\documentclass[a4paper,11pt]{kth-mag}
\usepackage[T1]{fontenc}
\usepackage{textcomp}
\usepackage{lmodern}
\usepackage[utf8]{inputenc}
\usepackage[swedish,english]{babel}
\usepackage{modifications}
\title{Anti-analysis techniques to preserve programmer anonymity in binary
executables}

\subtitle{
    DEGREE PROJECT IN COMPUTER SCIENCE, FIRST LEVEL \\
    STOCKHOLM, SWEDEN 2016
}
\author{Johan Wikström \\ Macaully Muir}
\date{April 2016}
\blurb{
    Supervisor: Michael Schliephake \\
    Examiner: Örjan Ekeberg
}
\trita{TRITA TK: xxx yyyy-nn}
\begin{document}
\frontmatter
\pagestyle{empty}
\removepagenumbers
\maketitle
\selectlanguage{english}
\begin{abstract}
Previous research has shown that it is possible to achieve authorship
attribution with high accuracy using binary files. The idea behind authorship
attribution for binaries is that authors’ unique style will survive the
compilation process making it possible to identify the author.    

The aim of this report is to investigate different ways to reduce the accuracy
and see how the processing of binaries affects the author prediction accuracy.

This work will build on the report of Rosenblum et al. and implement the method
they used. The method uses  a machine learning approach to predict the author
of the binary. The data used for prediction consists of features extracted from
the binaries. Similar to Rosenblum et al. (2011) binary files used in this
thesis come from the Google Code Jam programming competition.

Results from this study indicate that optimisation of code reduces the accuracy
achieved; however, is not enough to ensure anonymity. Using static linking
results in a more significant drop in accuracy. The data extracted shows that
optimisation and static linking results in more features that may be the cause
of the reduction in accuracy.
\end{abstract}
\clearpage
\begin{foreignabstract}{swedish}
I tidigare studier har det visat sig möjligt att med hög noggrannhet
identifiera författaren till binära filer. Idéen är att författare till kod
lämnar karaktäristiska drag i koden som går att känna igen även efter
kompilering. 

Syftet med denna rapport är att utforska olika tekniker för att minska
säkerheten då man vill bestämma vem som är författaren till binära filer och
undersöka hur olika bearbetningar av koden påverkar noggrannheten.     

Denna rapport bygger vidare på rapport av Rosenblum et al. och dess metod. En
maskininlärningsmetod implementerats för att identifiera vem som är författaren
till en binär fil. Attribut extraheras från binärfilerna som sedan kan används
som data för maskininlärning. Likt Rosenblum et al. används data hämtat från
programmeringstävlingen Google Code Jam. 

Erhållna resultat tyder på att optimering försämrar noggrannheten men är dock
inte tillräckligt för att garantera anonymitet. Användning av statisk länkning
resulterar i en kraftigare sänkning. Man kan se att optimering och statisk
länkning resulterar i fler attribut vilket kan vara orsaken till att det blir
svårare att avgöra vem som är författaren.
\end{foreignabstract}
\clearpage
\tableofcontents*
\mainmatter
\pagestyle{newchap}

\chapter{Introduction}
Is it possible to extract information about programming style from compiled
binary programs? Can such information be used to identify the author of a given
program? In 2011 Rosenblum et al. showed that the answer to both these
questions is “yes” under certain conditions (Rosenblum, 2011); from a pool of
191 Google Code Jam (GCJ) contestants with eight or more submitted solutions,
authorship identification from a set of 10 authors could be performed with 81\%
accuracy using a supervised support vector machine classifier. In 2015
Caliskan-Islam et al. proposed an alternate method that could achieve 96\%
accuracy from a pool of 20 contestants (Caliskan-Islam, 2015) under similar
conditions.

However, it is not clear whether the above results could be replicated with
other datasets from meaningful real-world scenarios. Nevertheless, they do
raise questions about the ability of programmers to write and distribute
software anonymously and it is therefore desirable to find methods that prevent
de-anonymisation with this kind of analysis. Using the methods of (Rosenblum
2011), this report aims to investigate how changes to the build process can be
used to reduce the accuracy of author classification on the same Google Code
Jam dataset.

\section{Problem definition}
In short, this thesis aims to:
\begin{itemize}
\item Validate the results of (Rosenblum, 2011) on the 2010 Google Code Jam dataset
\item Investigate how g++’s optimisation flags (O0-O3) affect the accuracy of the
      author classification algorithm on the 2010 GCJ dataset
\item Investigate how statically linking the standard library affects the accuracy of
      the author classification algorithm on the 2010 GCJ dataset.
\end{itemize}

\section{Motivation}
There are a number of legitimate reasons why the author of a piece of software
would want to remain anonymous. It could be that their association with the
software could put them in danger (for example, a programmer that writes
anti-surveillance software in an oppressive state). It could also be the case
that the programmer simply does not want to be publicly associated with the
software such as in the cases of TrueCrypt, Bitcoin etc. (TK: source). It is
therefore important to investigate the robustness of these author
classification algorithms and develop tools and procedures that ensure can help
ensure the anonymity of the programmer.

\section{Scope}
This report will only implement the method of Rosenblum et al. by using
available source code when possible and by implement remaining part ourselves
trying to be consistent with the original implementation.  Furthermore, we
limit our analysis to a subset of C++ solutions from the 2010 rounds of Google
Code Jam. This allows us to verify the results in the literature (Rosenblum,
2011) and provides a meaningful data set to work with. This method assume that
a program has exactly one author. However, this is often not true since many
programs are developed in teams.

\end{document}
